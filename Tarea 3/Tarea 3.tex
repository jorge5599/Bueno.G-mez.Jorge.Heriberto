\documentclass[12pt,a4paper]{article}
\usepackage[latin1]{inputenc}
\usepackage[spanish]{babel}
\usepackage{graphicx}
\usepackage{kpfonts}
\usepackage[left=2cm,right=2cm,top=2cm,bottom=2cm]{geometry}
\author{Jorge Bueno}
\begin{document}
\title{Universidad Politecnica \\ de la \\ Zona Metropolitana de Guadalajara}
\author{Tarea 3\\Jorge Heriberto Bueno Gomez 18312259\\Ing. Mecatronica 4 B}
\maketitle
$$\includegraphics[scale=.5]{UPCDLZMDG5783-logo.png} $$
\newpage
\section{Amplificador clase A} 
Son aquellos amplificador cuyas etapas de potencia consumen corrientes altas y continuas de su fuente de alimentacion, independientemente de si existe senal de audio o no.

\subsection{Caracteristicas}
Esta amplificacion presenta el inconveniente de generar una fuerte y constante emision de calor. No obstante, los transistores de salida estan siempre a una temperatura fija y sin alteraciones.

En general, se afirma que esta clase de amplificacion es frecuente en circuitos de audio y en los equipos domesticos de gama alta, ya que proporcionan una calidad de sonido potente y de muy buena calidad.

Los amplificador de clase A a menudo consisten en un transistor de salida conectado al positivo de la fuente de alimentacion y un transistor de corriente constante conectado de la salida al negativo de la fuente de alimentacion.

La senal del transistor de salida modula tanto el voltaje como la corriente de salida. Cuando no hay senal de entrada, la corriente de polarizacion constante fluye directamente del positivo de la fuente de alimentacion al negativo, resultando que no hay corriente de salida, se gasta mucha corriente. Algunos amplificador de clase A mas sofisticados tienen dos transistores de salida en configuracion push-pull.

\subsection{Ventajas}
La clase A se refiere a una etapa de salida con una corriente de polarizacion mayor que la maxima corriente de salida que dan, de tal forma que los transistores de salida siempre estan consumiendo corriente. La gran ventaja de la clase A es que es casi lineal, y en consecuencia la distorsion es menor.

\subsection{Desventaja}
La gran desventaja de la clase A es que es poco eficiente, se requiere un amplificador de clase A muy grande para dar 50 W, y ese amplificador usa mucha corriente y se pone a muy alta temperatura.

\subsection{Conclusion}
Los amplificadores de tipo A son se podria decir que es la clase mas fiel que se conoce ya que la senal que entra va a ser la misma que sale pero amplificada depende del voltaje que tenga,hay varios arreglos en los amplificadores, ya sea como cuando una senal se distorcione es porque la amplificacion excede del voltaje permitido y es cuando se corta y ya no es una onda senoidal, y se puede modificar ya sea disminuyendo el voltaje de entrada o aumentando el voltaje de la salida de la senal.
O tambien se puede alterar la senal porque una resistencia este mal y el transirtor tenga un voltaje menor y la senal se desfase.

\subsection{Referencias bibliograficas}
Amplificadores de potencia clase A: https://www.youtube.com/watchv=3FVnao-K2k\\
Amplificadores clase a: https://www.ecured.cu/AmplificadorclaseA
\end{document}