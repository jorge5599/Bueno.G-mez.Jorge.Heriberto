\documentclass[12pt,a4paper]{article}
\usepackage[utf8]{inputenc}
\usepackage[spanish]{babel}
\usepackage{graphicx}
\usepackage{kpfonts}
\usepackage[left=2cm,right=2cm,top=2cm,bottom=2cm]{geometry}
\author{Jorge Bueno}
\begin{document}
\title{Universidad Politecnica \\ de la \\ Zona Metropolitana de Guadalajara}
\author{Tarea 4\\Jorge Heriberto Bueno Gomez 18312259\\Ing. Mecatronica 4 B}
\maketitle
$$\includegraphics[scale=.5]{UPCDLZMDG5783-logo.png} $$
\newpage
\section{Amplificador clase B} 
Los amplificadores de clase B se caracterizan por tener intensidad casi nula a través de sus transistores cuando no hay señal en la entrada del circuito, por lo que en reposo el consumo es casi nulo. no.

\subsection{Caracteristicas}
Se les denomina amplificador clase B, cuando el voltaje de polarización y la máxima amplitud de la señal entrante poseen valores que hacen que la corriente de salida circule durante el semiciclo de la señal de entrada.

La característica principal de este tipo de amplificadores es el alto factor de amplificación.

Amplificadores clase AB: Estos básicamente son la mezcla de los dos anteriores. Cuando el voltaje de polarización y la máxima amplitud de la señal entrante poseen valores que hacen que la corriente de salida circule durante menos del ciclo completo y más de la mitad del ciclo de la señal de entrada, se les denomina: Amplificadores de potencia clase AB.

Dado que ocupa un lugar intermedio entre los de clase A y AB, cuando el voltaje de la señal es moderado funciona como uno de clase A, cuando la señal es fuerte se desempeña como uno de clase B, con una eficiencia y deformación moderadas.
Se les denomina amplificador clase B, cuando el voltaje de polarización y la máxima amplitud de la señal entrante poseen valores que hacen que la corriente de salida circule durante el semiciclo de la señal de entrada.

La característica principal de este tipo de amplificadores es el alto factor de amplificación.

Amplificadores clase AB: Estos básicamente son la mezcla de los dos anteriores. Cuando el voltaje de polarización y la máxima amplitud de la señal entrante poseen valores que hacen que la corriente de salida circule durante menos del ciclo completo y más de la mitad del ciclo de la señal de entrada, se les denomina: Amplificadores de potencia clase AB.

Dado que ocupa un lugar intermedio entre los de clase A y AB, cuando el voltaje de la señal es moderado funciona como uno dsultando que no hay corriente de salida, se gasta mucha corriente. Algunos amplificador de clase A mas sofisticados tienen dos transistores de salida en configuracion push-pull.

\subsection{Ventajas}
Posee bajo consumo en reposo.
Aprovecha al máximo la Corriente entregada por la fuente.
Intensidad casi nula cuando está en reposo.

\subsection{Desventaja}
Producen armónicos, y es mayor cuando no tienen los transistores de salida con las mismas características técnicas, debido a esto se les suele polarizar de forma que se les introduce una pequeña polarización directa. Con esto se consigue desplazar las curvas y se disminuye dicha distorsión.

\subsection{Aplicaciones}
Sistemas telefónicos,
Transmisores de seguridad portátiles
Sistemas de aviso, aunque no en audio.

\subsection{Conclusion}
El amplificador de clase B utilizan 2 transistores, tambien llamado push pull.

La máxima eficiencia en un amplificador clase B en contrafase es de 78.5 por ciento, por lo que un amplificador clase B en contrafase se utiliza mas comúnmente como etapa de salida que un amplificador de potencia clase A.

\subsection{Referencias bibliograficas}
Amplificadores de potencia clase B: https://www.youtube.com/watchv=3FVnao-K2k\\
Amplificadores clase a: https://www.ecured.cu/AmplificadorclaseB
\end{document}